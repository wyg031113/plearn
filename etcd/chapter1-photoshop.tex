
\mylineskip
\chapter{photoshop基本功能}\label{chaper:ch1}
\section{phtotshop简介}
Adobe Photoshop,简称“PS”,是由Adobe Systems开发和发行的图像处理软件。
Photoshop主要处理以像素所构成的数字图像。使用其众多的编修与绘图工具,可以
有效地进行图片编辑工作。PS有很多功能,在图像、图形、文字、视频、出版等各方
面都有涉及。2003年,Adobe Photoshop 8被更名为Adobe Photoshop CS。2013
年7月,Adobe公司推出了新版本的Photoshop CC,自此,Photoshop CS6作为Adobe
 CS系列的最后一个版本被新的CC系列取代。截止2016年12月Adobe PhotoshopCC2017
 所有数据类型见表\ref{tab:oftype}。至于详情可以参考~\cite{gzp}~和~\cite{nue}。
\oftypetable
 Adobe Photoshop,简称“PS”,是由Adobe Systems开发和发行的图像处理软件。
 Photoshop主要处理以像素所构成的数字图像。使用其众多的编修与绘图工具,可以
 有效地进行图片编辑工作。PS有很多功能,在图像、图形、文字、视频、出版等各方
 面都有涉及。2003年,Adobe Photoshop 8被更名为Adobe Photoshop CS。2013
 年7月,Adobe公司推出了新版本的Photoshop CC,自此,Photoshop CS6作为Adobe
 CS系列的最后一个版本被新的CC系列取代。截止2016年12月Adobe PhotoshopCC2017
 为市场最新版本。跳转到第\ref{chaper:ch1}章。\href{http://www.baidu.com}{这里是百度}后面还有字
 {\kaishu 这是楷体吗?}
 
  Adobe Photoshop,简称“PS”,是由Adobe Systems开发和发行的图像处理软件。
 Photoshop主要处理以像素所构成的数字图像。使用其众多的编修与绘图工具,可以
 有效地进行图片编辑工作。PS有很多功能,在图像、图形、文字、视频、出版等各方
 面都有涉及。2003年,Adobe Photoshop 8被更名为Adobe Photoshop CS。2013
 年7月,Adobe公司推出了新版本的Photoshop CC,自此,Photoshop CS6作为Adobe
 CS系列的最后一个版本被新的CC系列取代。截止2016年12月Adobe PhotoshopCC2017
 为市场最新版本。
 {\kaishu 这是楷体吗?}
\setlength{\abovedisplayskip}{0pt}
\setlength{\belowdisplayskip}{0pt}
 \begin{equation}\label{eq:fx}
 f(x)=3x^{2}+6(x-2)-1
 \end{equation}
 \begin{equation}\label{eq:gx}
g(x)=4x^{2}+6(x-8)+1
\end{equation}
 \begin{equation}\label{eq:zn}
E=mc^2
\end{equation}
  Adobe Photoshop,简称“PS”,是由Adobe Systems开发和发行的图像处理软件。
 Photoshop主要处理以像素所构成的数字图像。使用其众多的编修与绘图工具,可以
 有效地进行图片编辑工作。PS有很多功能,在图像、图形、文字、视频、出版等各方
 面都有涉及。2003年,Adobe Photoshop 8被更名为Adobe Photoshop CS。2013
 年7月,Adobe公司推出了新版本的Photoshop CC,自此,Photoshop CS6作为Adobe
 CS系列的最后一个版本被新的CC系列取代。截止2016年12月Adobe PhotoshopCC2017
 为市场最新版本。质能公式如公式\eqref{eq:zn}。
 {\kaishu 这是楷体吗?}
 \subsection{画布操作}
 \begin{enumerate}
 	\item 新建画布 \tab Ctrl + N
 	\item 画布切换 \tab F
 	\item 复位工作区 \tab Alt -> W -> K -> R
 	\item 放大缩小 \tab Alt + 鼠标滚轮(Ctrl + +, Ctrl + -)
 	\item 缩放工具 \tab Z(放大:鼠标点击画布,或按下Alt点击画布,可以放大缩小)
 	\item 移动画布 \tab Space + 鼠标左键按下拖动
 	\item 切换画布 \tab Ctrl + Tab
 	\item 显示网格 \tab Ctrl + '
 	\item 显示参考线 \tab Ctrl + ;
 	\item 显示标尺 \tab Ctrl + R
 \end{enumerate}
  Adobe Photoshop,简称“PS”,是由Adobe Systems开发和发行的图像处理软件。
 Photoshop主要处理以像素所构成的数字图像。使用其众多的编修与绘图工具,可以
 有效地进行图片编辑工作。PS有很多功能,在图像、图形、文字、视频、出版等各方
 面都有涉及。2003年,Adobe Photoshop 8被更名为Adobe Photoshop CS。2013
 年7月,Adobe公司推出了新版本的Photoshop CC,自此,Photoshop CS6作为Adobe
 CS系列的最后一个版本被新的CC系列取代。截止2016年12月Adobe PhotoshopCC2017
 为市场最新版本。
 {\kaishu 这是楷体吗?}
 
  Adobe Photoshop,简称“PS”,是由Adobe Systems开发和发行的图像处理软件。
 Photoshop主要处理以像素所构成的数字图像。使用其众多的编修与绘图工具,可以
 有效地进行图片编辑工作。PS有很多功能,在图像、图形、文字、视频、出版等各方
 面都有涉及。2003年,Adobe Photoshop 8被更名为Adobe Photoshop CS。2013
 年7月,Adobe公司推出了新版本的Photoshop CC,自此,Photoshop CS6作为Adobe
 CS系列的最后一个版本被新的CC系列取代。截止2016年12月Adobe PhotoshopCC2017
 为市场最新版本。
	\begin{figure}[!ht]
		\begin{center}
		\setlength{\belowcaptionskip}{10pt}
		\includegraphics[width=0.6\textwidth]{images/img1.jpg}
		\caption{风景1}\label{fig:s1}
		\end{center}
	\end{figure}
\myfig{风景2}{fig:s2}{0.6}{images/img2.jpg}
  Adobe Photoshop,简称“PS”,是由Adobe Systems开发和发行的图像处理软件。
 Photoshop主要处理以像素所构成的数字图像。使用其众多的编修与绘图工具,可以
 有效地进行图片编辑工作。PS有很多功能,在图像、图形、文字、视频、出版等各方
 面都有涉及。2003年,Adobe Photoshop 8被更名为Adobe Photoshop CS。2013
 年7月,Adobe公司推出了新版本的Photoshop CC,自此,Photoshop CS6作为Adobe
 CS系列的最后一个版本被新的CC系列取代。截止2016年12月Adobe PhotoshopCC2017
 为市场最新版本。如图\ref{fig:s1}所示,又如图\ref{fig:s2}所示。
 {\kaishu 这是楷体吗?}
 
  Adobe Photoshop,简称“PS”,是由Adobe Systems开发和发行的图像处理软件。
 Photoshop主要处理以像素所构成的数字图像。使用其众多的编修与绘图工具,可以
 有效地进行图片编辑工作。PS有很多功能,在图像、图形、文字、视频、出版等各方
 面都有涉及。2003年,Adobe Photoshop 8被更名为Adobe Photoshop CS。2013
 年7月,Adobe公司推出了新版本的Photoshop CC,自此,Photoshop CS6作为Adobe
 CS系列的最后一个版本被新的CC系列取代。截止2016年12月Adobe PhotoshopCC2017
 为市场最新版本。
 {\kaishu 这是楷体吗?}
 
  Adobe Photoshop,简称“PS”,是由Adobe Systems开发和发行的图像处理软件。
 Photoshop主要处理以像素所构成的数字图像。使用其众多的编修与绘图工具,可以
 有效地进行图片编辑工作。PS有很多功能,在图像、图形、文字、视频、出版等各方
 面都有涉及。2003年,Adobe Photoshop 8被更名为Adobe Photoshop CS。2013
 年7月,Adobe公司推出了新版本的Photoshop CC,自此,Photoshop CS6作为Adobe
 CS系列的最后一个版本被新的CC系列取代。截止2016年12月Adobe PhotoshopCC2017
 为市场最新版本。
 {\kaishu 这是楷体吗?}
 
\subsection{phtotshop简介}
 \newpage