\mylineskip
\chapter{ETCD应用场景}\label{chaper:ch1}
\section{ETCD集群}
ETCD从集群必须是奇数个结点,每个结点要监听两个端口,一个是用于接收客户端请求(client-urls),另一个用于接收集群中其他ETCD服务器的请求(peer-urls)。
可以用docker或者多个机器来部署。每个机器一个ETCD服务器。这里用于测试,就在一个机器的不同端口上启动了三个ETCD服务器。
\begin{minted}[linenos,firstnumber=1, fontfamily=tt, baselinestretch=0.7, numberblanklines=false, style=vs,xleftmargin=20pt,breaklines]{bash}
#启动三个etcd服务器组成一个集群
#可以把下面的脚本放在
#root/go/src/go.etcd.io/etcd/etcd_cluster目录下。然后执行,就能构建出一个集群了。其中/root/go是GOPATH目录。
TOKEN=token-01
CLUSTER_STATE=new
NAME_1=m1
NAME_2=m2
NAME_3=m3
#在同一个机器可以用不同的端口
HOST_1=127.0.0.1
HOST_2=127.0.0.1
HOST_3=127.0.0.1

CLIENT_PORT_1=2379
PEER_PORT_1=2380
CLIENT_PORT_2=2381
PEER_PORT_2=2382
CLIENT_PORT_3=2383
PEER_PORT_3=2384

#data目录用来存放结点数据
DATA1=data1
DATA2=data2
DATA3=data3
CLUSTER=${NAME_1}="http://${HOST_1}:$PEER_PORT_1,\
${NAME_2}=http://${HOST_2}:$PEER_PORT_2,\
${NAME_3}=http://${HOST_3}:$PEER_PORT_3"

#分别启动三个结点
$etcd --data-dir=$DATA1 --name ${NAME_1} \
--initial-advertise-peer-urls http://${HOST_1}:$PEER_PORT_1 \
--listen-peer-urls http://$HOST_1:$PEER_PORT_1 \
--advertise-client-urls http://$HOST_1:$CLIENT_PORT_1 \
--listen-client-urls http://${HOST_1}:$CLIENT_PORT_1 \
--initial-cluster ${CLUSTER} \
--initial-cluster-state ${CLUSTER_STATE} \
--initial-cluster-token ${TOKEN} &

$etcd --data-dir=$DATA2 --name ${NAME_2} \
--initial-advertise-peer-urls http://${HOST_2}:$PEER_PORT_2 \
--listen-peer-urls http://$HOST_2:$PEER_PORT_2 \
--advertise-client-urls http://$HOST_2:$CLIENT_PORT_2 \
--listen-client-urls http://${HOST_2}:$CLIENT_PORT_2 \
--initial-cluster ${CLUSTER} \
--initial-cluster-state ${CLUSTER_STATE} \
--initial-cluster-token ${TOKEN} &

$etcd --data-dir=$DATA3 --name ${NAME_3} \
--initial-advertise-peer-urls http://${HOST_3}:$PEER_PORT_3 \
--listen-peer-urls http://$HOST_3:$PEER_PORT_3 \
--advertise-client-urls http://$HOST_3:$CLIENT_PORT_3 \
--listen-client-urls http://${HOST_3}:$CLIENT_PORT_3 \
--initial-cluster ${CLUSTER} \
--initial-cluster-state ${CLUSTER_STATE} \
--initial-cluster-token ${TOKEN} &

#ENDPONTS是给etcdctl用的
export ETCDCTL_API=3
export ENDPOINTS="$HOST_1:$CLIENT_PORT_1,\
$HOST_2:$CLIENT_PORT_2,\
$HOST_3:$CLIENT_PORT_3"

echo waiting for etcd server cluster init ...
sleep 5
echo
echo List all members in cluster:
$etcdctl --endpoints=$ENDPOINTS member list
echo
echo client used endpoints:
echo "$ENDPOINTS"


\end{minted}