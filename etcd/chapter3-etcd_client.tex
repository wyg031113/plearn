\mylineskip
\chapter{ETCD应用场景}\label{chaper:ch1}
\section{ETCD集群}
ETCD从集群必须是奇数个结点,每个结点要监听两个端口,一个是用于接收客户端请求(client-urls),另一个用于接收集群中其他ETCD服务器的请求(peer-urls)。
可以用docker或者多个机器来部署。每个机器一个ETCD服务器。这里用于测试,就在一个机器的不同端口上启动了三个ETCD服务器。
\begin{minted}[linenos,firstnumber=1, fontfamily=tt, baselinestretch=0.7, numberblanklines=false, style=vs,xleftmargin=20pt,breaklines]{bash}
#启动三个etcd服务器组成一个集群
#可以把下面的脚本放在
#root/go/src/go.etcd.io/etcd/etcd_cluster目录下。然后执行,就能构建出一个集群了。其中/root/go是GOPATH目录。
TOKEN=token-01
CLUSTER_STATE=new
NAME_1=m1
NAME_2=m2
NAME_3=m3
#在同一个机器可以用不同的端口
HOST_1=127.0.0.1
HOST_2=127.0.0.1
HOST_3=127.0.0.1

CLIENT_PORT_1=2379
PEER_PORT_1=2380
CLIENT_PORT_2=2381
PEER_PORT_2=2382
CLIENT_PORT_3=2383
PEER_PORT_3=2384

#data目录用来存放结点数据
DATA1=data1
DATA2=data2
DATA3=data3
CLUSTER=${NAME_1}="http://${HOST_1}:$PEER_PORT_1,\
${NAME_2}=http://${HOST_2}:$PEER_PORT_2,\
${NAME_3}=http://${HOST_3}:$PEER_PORT_3"

#分别启动三个结点
$etcd --data-dir=$DATA1 --name ${NAME_1} \
--initial-advertise-peer-urls http://${HOST_1}:$PEER_PORT_1 \
--listen-peer-urls http://$HOST_1:$PEER_PORT_1 \
--advertise-client-urls http://$HOST_1:$CLIENT_PORT_1 \
--listen-client-urls http://${HOST_1}:$CLIENT_PORT_1 \
--initial-cluster ${CLUSTER} \
--initial-cluster-state ${CLUSTER_STATE} \
--initial-cluster-token ${TOKEN} &

$etcd --data-dir=$DATA2 --name ${NAME_2} \
--initial-advertise-peer-urls http://${HOST_2}:$PEER_PORT_2 \
--listen-peer-urls http://$HOST_2:$PEER_PORT_2 \
--advertise-client-urls http://$HOST_2:$CLIENT_PORT_2 \
--listen-client-urls http://${HOST_2}:$CLIENT_PORT_2 \
--initial-cluster ${CLUSTER} \
--initial-cluster-state ${CLUSTER_STATE} \
--initial-cluster-token ${TOKEN} &

$etcd --data-dir=$DATA3 --name ${NAME_3} \
--initial-advertise-peer-urls http://${HOST_3}:$PEER_PORT_3 \
--listen-peer-urls http://$HOST_3:$PEER_PORT_3 \
--advertise-client-urls http://$HOST_3:$CLIENT_PORT_3 \
--listen-client-urls http://${HOST_3}:$CLIENT_PORT_3 \
--initial-cluster ${CLUSTER} \
--initial-cluster-state ${CLUSTER_STATE} \
--initial-cluster-token ${TOKEN} &

#ENDPONTS是给etcdctl用的
export ETCDCTL_API=3
export ENDPOINTS="$HOST_1:$CLIENT_PORT_1,\
$HOST_2:$CLIENT_PORT_2,\
$HOST_3:$CLIENT_PORT_3"

echo waiting for etcd server cluster init ...
sleep 5
echo
echo List all members in cluster:
$etcdctl --endpoints=$ENDPOINTS member list
echo
echo client used endpoints:
echo "$ENDPOINTS"

\end{minted}

\section{ETCD客户端命令}
ETCD自还了一个etcdctl客户端工具。用它来熟悉基本功能十分好。
\subsection{设置一些常用变量}
\begin{minted}[linenos,firstnumber=1, fontfamily=tt, baselinestretch=0.7, numberblanklines=false, style=vs,xleftmargin=20pt,breaklines]{bash}
EP=127.0.0.1:2379,127.0.0.1:2381,127.0.0.1:2383
alias ec="../bin/etcdctl --endpoints=$EP"
cd ~/go/src/go.etcd.io/etcd/etcd_cluster
#~/go为GOPATH指定目录
\end{minted}

\subsection{基本的增删改查(get/put/del)}
\begin{minted}[linenos,firstnumber=1, fontfamily=tt, baselinestretch=0.7, numberblanklines=false, style=vs,xleftmargin=20pt,breaklines]{bash}
../bin/etcdctl --endpoints=$EP put foo "Hello, World"
../bin/etcdctl --endpoints=$EP get foo
../bin/etcdctl --endpoints=$EP --write-out="json" get foo
\end{minted}

\subsection{带前缀的查删(--prefix)}
\begin{minted}[linenos,firstnumber=1, fontfamily=tt, baselinestretch=0.7, numberblanklines=false, style=vs,xleftmargin=20pt,breaklines]{bash}
#带前缀的查找:
../bin/etcdctl --endpoints=$EP put web1 value1
../bin/etcdctl --endpoints=$EP put web2 value2
../bin/etcdctl --endpoints=$EP put web3 value3
../bin/etcdctl --endpoints=$EP get web --prefix
#输出:
web1
value1
web2
value2
web3
value3
#带前缀的删除:
../bin/etcdctl --endpoints=$EP put key myvalue
../bin/etcdctl --endpoints=$EP put k1 value1
../bin/etcdctl --endpoints=$EP put k2 value2
#删除所有以k开头的键
../bin/etcdctl --endpoints=$EP del k --prefix
\end{minted}

\subsection{事务(txn)}
\begin{minted}[linenos,firstnumber=1, fontfamily=tt, baselinestretch=0.7, numberblanklines=false, style=vs,xleftmargin=20pt,breaklines]{bash}
#实现类型下面代码的功能
#if value("user1") = "bad" :
#    del user1
#else:
#    put user1 good
ec put user1 bad
ec txn --interactive
compares:
value("user1") = "bad"

success requests (get, put, del):
del user1

failure requests (get, put, del):
put user1 good

SUCCESS

1
#上边最终会执行del user1
\end{minted}

\subsection{监听(watch)}
\begin{minted}[linenos,firstnumber=1, fontfamily=tt, baselinestretch=0.7, numberblanklines=false, style=vs,xleftmargin=20pt,breaklines]{bash}
#需要开两个终端
#终端1执行下面命令会阻塞
ec watch stock --prefix
#终端2执行
EP=127.0.0.1:2379,127.0.0.1:2381,127.0.0.1:2383
alias ec="../bin/etcdctl --endpoints=$EP"
ec put stock1 10
ec put stock2 20
#终端1上会显示
PUT
stock1
10
PUT
stock2
20
\end{minted}

\subsection{租约(lease)}
租约可以认为是一个会定时删除的对象,这个对象超时删除时,会把与它绑定的所有key都连带删除。
租约可以不断续租,也可以立即回收。
\begin{minted}[linenos,firstnumber=1, fontfamily=tt, baselinestretch=0.7, numberblanklines=false, style=vs,xleftmargin=20pt,breaklines]{bash}
#分配lease
lease=`ec lease grant 5|awk '{print $2}'`
echo "Lease:" $lease

#设置k v时关联一个lease
ec put sample value --lease=$lease
echo "Get1:"
ec get sample

#保持lease存在
ec lease keep-alive $lease & 
#或者可以立即收回lease
#ec lease lease revoke $lease
sleep 10

#依然可以获得
echo "Get2:"
ec get sample

#停止保持
pkill etcdctl

#5秒之后就自动删除了
echo "Sleep 6 seconds"
sleep 6
echo "Get3:"
ec get sample
\end{minted}


\subsection{分布式锁(lock)}
\begin{minted}[linenos,firstnumber=1, fontfamily=tt, baselinestretch=0.7, numberblanklines=false, style=vs,xleftmargin=20pt,breaklines]{bash}
#term1执行:
ec lock mutex1
mutex1/93e6f5a8fafa72c

#打印出这个表示获得了锁,并阻塞在这里了
#term2执行:
ec lock mutex1
#直接阻塞了,因为锁被term1占用了

#term1 CTRL+C停止,释放了锁
^C

#term2此时输出:
mutex1/93e6f5a8fafa72f #表示获得了锁,并阻塞在这里,一直战胜
\end{minted}

\subsection{选主(elect)}
\begin{minted}[linenos,firstnumber=1, fontfamily=tt, baselinestretch=0.7, numberblanklines=false, style=vs,xleftmargin=20pt,breaklines]{bash}
#ETCD多个服务器组成集群,其中要用一个来当Leader。也就是用RAFT算法选出Leader。
ec elect one p1 &
ec elect one p2 &
\end{minted}

\subsection{查看集群状态(status/health)}
\begin{minted}[linenos,firstnumber=1, fontfamily=tt, baselinestretch=0.7, numberblanklines=false, style=vs,xleftmargin=20pt,breaklines]{bash}
#查看集群状态
ec --write-out=table endpoint status

将会输出:
 ENDPOINT|ID|VERSION| DB SIZE | IS LEADER | IS LEARNER | 
 RAFT TERM | RAFT INDEX | RAFT APPLIED INDEX | ERRORS 
 
ec --write-out=table endpoint healt
#将会输出:
ENDPOINT    | HEALTH |    TOOK    | ERROR
\end{minted}

\subsection{生成快照(snapshot)}
\begin{minted}[linenos,firstnumber=1, fontfamily=tt, baselinestretch=0.7, numberblanklines=false, style=vs,xleftmargin=20pt,breaklines]{bash}
#生成快照
../bin/etcdctl --endpoints=$ENDPOINT_ONE snapshot save my.db
#查看快照
../bin/etcdctl --write-out=table --endpoints=$ENDPOINT_ONE snapshot status my.db
\end{minted}

\subsection{集群添加删除成员(member remove/add)}
删除老的成员,加一个新的成员,而不是把老成员加回来。
\begin{minted}[linenos,firstnumber=1, fontfamily=tt, baselinestretch=0.7, numberblanklines=false, style=vs,xleftmargin=20pt,breaklines]{bash}
ec --write-out=table member list

#得到m2的member id
m2_id=`ec member list|grep m2|awk -F ',' '{print $1}'`

#删除member
ec member remove $m2_id
echo Sleep to wait removing $m2_id finish...
sleep 5

#添加新结点
ec --write-out=table member list
../bin/etcdctl --endpoints=127.0.0.1:2379,127.0.0.1:2383 member add m4 \
--peer-urls=http://127.0.0.1:2386 

echo "Sleep to wait add new member finish..."
sleep 5

#启动新结点,注册cluster和initial-cluster-state变了
../bin/etcd --data-dir=data4 --name m4 \
--initial-advertise-peer-urls http://127.0.0.1:2386 \
--listen-peer-urls http://127.0.0.1:2386 \
--advertise-client-urls http://127.0.0.1:2385 \
--listen-client-urls http://127.0.0.1:2385 \
--initial-cluster "m1=http://127.0.0.1:2380,\
m4=http://127.0.0.1:2386,m3=http://127.0.0.1:2384" \
--initial-cluster-state existing \
--initial-cluster-token token-01 &

echo "Sleep to wait new node starting..."
sleep 5
ec --write-out=table member list
\end{minted}

\subsection{认证(auth)}
删除老的成员,加一个新的成员,而不是把老成员加回来。
\begin{minted}[linenos,firstnumber=1, fontfamily=tt, baselinestretch=0.7, numberblanklines=false, style=vs,xleftmargin=20pt,breaklines]{bash}
#添加角色
ec role add root
#给角色授权
ec role grant-permission root readwrite foo
#查看角色
ec role get root
#必须有一个叫root的user
ec user add root
#让用户担任某角色
ec user grant-role root root
#启用认证
ec auth enable
#读写
ec --user=root:root put foo bar
ec --user=root:root get foo
#停止认证
ec auth disable --user=root:root
\end{minted}
