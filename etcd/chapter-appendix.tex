\chapter{附录}
\mylineskip
%minted使用参考ftp://ftp.dante.de/tex-archive/macros/latex/contrib/minted/minted.pdf
{\noindent C语言结构体用法}
\usemintedstyle[c++]{vs} %查看支持的样式:命令行中执行pygmentize -L styles; 支持的语言: pygmentize -L lexers
\renewcommand{\theFancyVerbLine}{\sffamily\textcolor[rgb]{0.6,0.2,0.3}{\normalsize\oldstylenums{\arabic{FancyVerbLine}}}}
\begin{minted}[linenos,firstnumber=11, fontfamily=tt, baselinestretch=0.7, numberblanklines=false, style=vs,xleftmargin=20pt]{c++}

#include <stdio.h>
#include <string.h>

struct Books
{
    char  title[50];
    char  author[50];
    char  subject[100];
    int   book_id;
};

int main( )
{
    struct Books Book1;        /* 声明 Book1,类型为 Books */
    struct Books Book2;        /* 声明 Book2,类型为 Books */
    
    /* Book1 详述 */
    strcpy( Book1.title, "C Programming");
    strcpy( Book1.author, "Nuha Ali"); 
    strcpy( Book1.subject, "C Programming Tutorial");
    Book1.book_id = 6495407;
    
    /* Book2 详述 */
    strcpy( Book2.title, "Telecom Billing");
    strcpy( Book2.author, "Zara Ali");
    strcpy( Book2.subject, "Telecom Billing Tutorial");
    Book2.book_id = 6495700;
    
    /* 输出 Book1 信息 */
    printf( "Book 1 title : %s\n", Book1.title);
    printf( "Book 1 author : %s\n", Book1.author);
    printf( "Book 1 subject : %s\n", Book1.subject);
    printf( "Book 1 book_id : %d\n", Book1.book_id);
    
    /* 输出 Book2 信息 */
    printf( "Book 2 title : %s\n", Book2.title);
    printf( "Book 2 author : %s\n", Book2.author);
    printf( "Book 2 subject : %s\n", Book2.subject);
    printf( "Book 2 book_id : %d\n", Book2.book_id);
    
    return 0;
}
\end{minted}

{\noindent 示例2 C\#代码}
\begin{minted}[mathescape,
    linenos,
    numbersep=5pt,
    gobble=2,
    frame=lines,
    framesep=2mm]{csharp}
    string title = "This is a Unicode π in the sky"
    /*
    Defined as $\pi=\lim_{n\to\infty}\frac{P_n}{d}$ where $P$ is the perimeter
    of an $n$-sided regular polygon circumscribing a
    circle of diameter $d$.
    */
    const double pi = 3.1415926535
\end{minted}


%使用listings
\lstset{
    columns=fixed,       
    numbers=left,                                        % 在左侧显示行号
    numberstyle=\tiny\color{gray},                       % 设定行号格式
    frame=none,                                          % 不显示背景边框
    backgroundcolor=\color[RGB]{245,245,244},            % 设定背景颜色
    keywordstyle=\color[RGB]{40,40,255},                 % 设定关键字颜色
    numberstyle=\footnotesize\color{darkgray},           
    commentstyle=\it\color[RGB]{0,96,96},                % 设置代码注释的格式
    stringstyle=\rmfamily\slshape\color[RGB]{128,0,0},   % 设置字符串格式
    showstringspaces=false,                              % 不显示字符串中的空格
    language=c++,                                        % 设置语言
}
\begin{lstlisting}[language={c++}]
#include  <stdio.h>

int main(){
    int  sum=0;
    int  num=1;
    int  sum2=0;
    int  num2=2;
    while(num<100){
        sum=sum+num;
        num=num+2;
    }
    printf("奇数和为:%d\n",sum);
    
    while(num2<=100){
        sum2=sum2+num2;
        num2=num2+2;
    }
    printf("偶数和为:%d\n",sum2);
}
\end{lstlisting}

\inputminted[breaklines=true]{c}{source/01.c}
