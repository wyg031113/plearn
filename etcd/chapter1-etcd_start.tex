
\mylineskip
\chapter{etcd编译安装}\label{chaper:ch1}
\section{golang环境}
当然在安装etcd前要先安装go。并且设置好GOPATH。可以用apt,yum或者源码安装,下载二进制安装等方式。
从官网下载golang源码包。https://golang.org/dl/。
\myfig{golang环境变量}{fig:s1}{0.9}{images/golang_env.png}

\begin{minted}[linenos,firstnumber=1, fontfamily=tt, baselinestretch=0.7, numberblanklines=false, style=vs,xleftmargin=10pt]{shell}
wget https://golang.org/doc/install?download=go1.13.5.linux-amd64.tar.gz
tar xvf go1.13.5.linux-amd64.tar.gz -C /usr/local
并在vim ~/.bashrc加入
export PATH=$PATH:/user/local/go/bin
\end{minted}


\section{etcd编译安装}
\begin{minted}[linenos,firstnumber=1, fontfamily=tt, baselinestretch=0.7, numberblanklines=false, style=vs,xleftmargin=10pt]{shell}
#etcd编译
mkdir -p  $GOPATH/src/go.etcd.io/
cd $GOPATH/src/go.etcd.io/
git clone https://github.com/etcd-io/etcd.git
./build
./bin/etcd
\end{minted}

\myfig{etcd运行}{fig:s2}{0.9}{images/etcd_starting.png}

\section{etcd编译文件build分析}
	etcd的build文件如图\ref{fig:s3}所示。 golang的编译非常简洁快速。直接编译出了etcd和etcdctl两个可执行文件。

\myfig{etcd build}{fig:s3}{0.9}{images/etcd_build.png}

\section{etcd版本}
	本书开写时的etcd的最新版本。如果看代码,要用与本书一致比较好。
\myfig{etcd版本}{fig:s4}{0.9}{images/etcd_version.png}

\section{学习目标与方法}
学习ETCD是学的什么?怎么学?当然是etcd的知识。在学习过程中要多问自己问题,比如ETCD启动流程是什么?ETCD是如何处理客户端
请求的? ETCD提供了什么服务?
问题 --> 分析源码 --> 答案
问题与答案便是监督学习的训练集。 分析源码是有方法的,通过各种有效方法分析才能尽快的找到答案。
那么,如何分析源码又转化问题,他的答案呢?

经过训练,大脑里会形成一个关于ETCD问题的模型,你以为你很了解这个模型,其实不然,至于大脑中形成了什么样的模型,目前无法知道。


如何知道自己学会了没。方法1:讲出来,做笔记,或者讲给自己听。(解释法) 方法2:网上查:ETCD面试题,有很多面试题,看看自己能
答出不?(测试集) 很不幸的是测试集一旦看过一遍,就会变成训练集。 除非看了之后,不要记住,忘掉。 方法3:应用到实践当中去。

\section{ETCD应用}
\subsection{服务发现}
\subsection{配置管理}
\subsection{集群选主}
\subsection{分布式锁}
\subsection{发布定阅}
\subsection{负载均衡}
\subsection{分布式命名}
\subsection{ETCD高可用集群搭建}

