
\mylineskip
\chapter{etcd编译安装}\label{chaper:ch1}
\section{golang环境}
当然在安装etcd前要先安装go。并且设置好GOPATH。可以用apt,yum或者源码安装,下载二进制安装等方式。
从官网下载golang源码包。https://golang.org/dl/。
\myfig{golang环境变量}{fig:s1}{0.9}{images/golang_env.png}

\begin{minted}[linenos,firstnumber=1, fontfamily=tt, baselinestretch=0.7, numberblanklines=false, style=vs,xleftmargin=10pt]{shell}
wget https://golang.org/doc/install?download=go1.13.5.linux-amd64.tar.gz
tar xvf go1.13.5.linux-amd64.tar.gz -C /usr/local
并在vim ~/.bashrc加入
export PATH=$PATH:/user/local/go/bin
\end{minted}


\section{etcd编译安装}
\begin{minted}[linenos,firstnumber=1, fontfamily=tt, baselinestretch=0.7, numberblanklines=false, style=vs,xleftmargin=10pt]{shell}
#etcd编译
mkdir -p  $GOPATH/src/go.etcd.io/
cd $GOPATH/src/go.etcd.io/
git clone https://github.com/etcd-io/etcd.git
./build
./bin/etcd
\end{minted}

\myfig{etcd运行}{fig:s2}{0.9}{images/etcd_starting.png}

\section{etcd编译文件build分析}
	etcd的build文件如图\ref{fig:s3}所示。 golang的编译非常简洁快速。直接编译出了etcd和etcdctl两个可执行文件。

\myfig{etcd build}{fig:s3}{0.9}{images/etcd_build.png}

\section{etcd版本}
	本书开写时的etcd的最新版本。如果看代码,要用与本书一致比较好。
\myfig{etcd版本}{fig:s4}{0.9}{images/etcd_version.png}
